\import{Oppgaver/}{Oppgave1a.tex}


I beregningene vil følgende symboler bli brukt: (sett inn symbolene og hva deres benevning er. Gjelder for alle ligninger om ikke annet er spesifisert for den enkle ligningen. Eks:konvertering av masse til mol.)

\subsection{Valg av basis og konvertering mellom ulike basis}

I denne oppgaven ble det brukt massebasis med benevningen Kg. En massebasis ble valgt fordi innsrømen til prossesen er oppgitt i Kg. Konverteringen mellom Kg og Mol gjøres med følgende formel:

I denne lingnigen står m for masse gitt i gram, n står for Mol og M står for molar masse gitt ved g/mol. Denne ligningen er hentet fra ...?


\subsection{Massebalanser og energibalanser for strømmene i prosessen}
Vi har gjort noen antake


\subsubsection{Massebalanser}

Følgende ligninger ble brukt for å 

\subsubsection{Energibalanser}

\subsection{Energiforbruk i stripper}

\subsection{Varmeveksling i prosessen}

\subsection{Kompresjon av produktstrøm}

\subsubsection{Kompresjon i tre steg}

\subsubsection{Kompresjon i ett steg}

