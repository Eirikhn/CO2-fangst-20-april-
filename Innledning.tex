Utslipp av CO\textsubscript{2} er den gassen som er med på å forandre klimaet på jorden mest. Det er dessverre umulig å unngå at man får dannet CO\textsubscript{2} som et biprodukt i veldig mange tilfeller som for eksempel utvinning av jern eller aluminium. Det er derfor et større fokus på å få lagret denne CO\textsubscript{2}en i store lagre. For å gjøre dette ønsker man å ha en så ren strøm av CO\textsubscript{2}-gass som mulig. To av de mest brukte metodene for å splitte CO\textsubscript{2} fra det man kaller inerte gasser kalles for post-combustion CO\textsubscript{2}-fangst og pre-combustion CO\textsubscript{2}-fangst. Denne rapporten omhandler post-combustion CO\textsubscript{2}-fangst og vil ta for seg prosessene og de viktige hovedmomentene for denne prosessen. Det vil også bli lagt vekt på muligheter for å spare energi siden dette er en prosess som krever mye energi. 

Selve prosessen går ut på å absorbere CO\textsubscript{2} i en MEA-løsning for så å skille disse ved en senere anledning i en stripper slik at man får to utstrømmer der den ene består av inerte gasser og noe CO\textsubscript{2}, mens den andre består av utelukkende CO\textsubscript{2} i gassform.

Rapporten er utarbeidet som et prosjekt for faget TKP4120 ved NTNU. Hensikten med rapporten er å få kunnskap om en relevant prosess som brukes i prosessanlegg i dag og om dens samfunnsrelevans. 