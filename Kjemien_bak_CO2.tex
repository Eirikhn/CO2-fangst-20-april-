\begin{equation}
    \ce{CO_{2}(g) <=> CO_{2}(aq)}
\end{equation}

Vi går ut i fra at alt CO\textsubscript{2} enten blir absorbert eller løst i blandingen (altså at mengden CO\textsubscript{2}(g) er like 0 etter absorberen).

\begin{equation}
    \ce{CO_{2}(aq)+2HO(CH_{2})_{2}NH_{2}(aq) <=> HO(CH_{2})_{2}NH_{3}^{+}(aq) + HO(CH_{2})_{2}NHCOO^{-}(aq)}
\end{equation}

Loadingen er gitt ved: 

\begin{equation}
    \ce{\alpha = \frac{[CO_{2}]_{absorbert}}{[MEA]_{0}}}
\end{equation}

Henrys lov er gitt ved:

\begin{equation}
    \ce{p_{CO_{2}}= K_{H}[CO_{2}]}
\end{equation}


%Første oppgaven som omhandler bevis for en eller annen ting
\import{Oppgaver/}{Oppgave1a.tex}

%Andre oppgaven der alpha øker:
\import{Oppgaver/}{Oppgave1b.tex}

%Tredje oppgaven der temperaturen minker: 
\import{Oppgaver/}{Oppgave1c.tex}

%Fjerde del av oppgaven der forklaringer ligger
\import{Oppgaver/}{Oppgave1d.tex}

%Konklusjon
\import{Oppgaver/}{KonklusjonDel1.tex}